\documentclass[../main.tex]{subfiles}
\begin{document}
\begin{center}
		{\bfseries\Large \textsf{List of Abbreviations}}
\end{center}
\addcontentsline{toc}{chapter}{List of Abbreviations}
\begin{table}[h!]
\def\arraystretch{2.0}
\begin{tabularx}{\textwidth}{YY}
Abbreviation & Meaning \\
\hline
ACSF	&	Atomic-centred symmetry functions	\\
AE	&	Absolute error	\\
AFIR	&	Artificial force induced reaction	\\
AIMD	&	Ab initio molecular dynamics	\\
AINR	&	Ab initio nanoreactor	\\
AL	&	Active learning	\\
ANN	&	Artificial neural network	\\
AO	&	Atomic orbtial	\\
API	&	Application programming interface	\\
BFGS	&	Broyden-Fletcher-Goldfarb-Shanno optimisation	\\
BSIE	&	Basis set incompleteness error	\\
BSSE	&	Basis set superposition error	\\
CC	&	Coupled cluster	\\
CCDC	&	Cambridge Crystallographic Data Centre	\\
CCSD	&	Coupled cluster singles and doubles	\\
CCSD(T)	&	CCSD with perturbative triples	\\
CI	&	Configuration interaction	\\
COM	&	Centre of mass	\\
DA	&	Diels-Alder	\\
\end{tabularx}
\end{table}
\newpage
\begin{table}[h!]
\def\arraystretch{2.0}
\begin{tabularx}{\textwidth}{YY}
Abbreviation & Meaning \\
\hline
DCM	&	Dichloromethane	\\
DFT	&	Density Functional Theory	\\
DLPNO	&	Domain-based local pair natural orbital	\\
ECP	&	Effective core potential	\\
EDDM	&	Electron density difference map	\\
ESP	&	Electrostatic potential	\\
ETDG	&	Experimental-Torsion Distance Geometry	\\
ETKDG	&	Experimental-Torsion Distance Geometry with Knowlege	\\
EW	&	Exponential well	\\
FEP	&	Free energy perturbation	\\
FF	&	Force field	\\
FSM	&	Freezing string method	\\
GAP	&	Gaussian approximation potential	\\
GDML	&	Gradient-domain machine learning	\\
GGA	&	Generalized gradient approximation	\\
GPR	&	Gaussian process regression	\\
GSM	&	Growing string method	\\
GTO	&	Gaussian-type orbtial	\\
GUI	&	Graphical user interface	\\
\end{tabularx}
\end{table}
\newpage
\begin{table}[h!]
\def\arraystretch{2.0}
\begin{tabularx}{\textwidth}{YY}
Abbreviation & Meaning \\
\hline
HF	&	Hartree-Fock	\\
HO	&	Harmonic oscillator	\\
HOMO	&	Highest occupied molecular orbtial	\\
HVTS	&	High-throughput virtual screening	\\
IGM	&	Ideal gas method	\\
KS	&	Kohn-Sham	\\
LCC	&	Local coupled cluster	\\
LEPS	&	London-Eyring-Polanyi-Sato	\\
LFM	&	Low frequency mode	\\
LUMO	&	Lowest unoccupied molecular orbtial	\\
MAD	&	Mean absolute deviation	\\
MC	&	(Metropolis) Monte Carlo	\\
MD	&	Molecular dynamics	\\
MEP	&	Minimum energy pathway	\\
ML	&	Machine learning	\\
MLP	&	Machine learned potential	\\
MO	&	Molecular orbtial	\\
MOF	&	Metal organic framework	\\
MP2	&	M{\o}ller-Plesset perturbation theory to 2nd order	\\
\end{tabularx}
\end{table}
\newpage
\begin{table}[h!]
\def\arraystretch{2.0}
\begin{tabularx}{\textwidth}{YY}
Abbreviation & Meaning \\
\hline
MSD	&	Mean signed deviation	\\
MSE	&	Mean squared error	\\
NBO	&	Natural bond orbtial	\\
NCI	&	Non-covalent interaction	\\
NEB	&	Nudged elastic band	\\
NN	&	[Artificial] neural network	\\
PES	&	Potential energy surface	\\
PF	&	Partition function	\\
PIB	&	Particle in a box	\\
PP	&	Pseudopotential	\\
PT	&	Perturbation theory	\\
QM	&	Quantum mechanics	\\
QM/MM	&	Quantum mechanics in a molecular mechanics environment	\\
QSAR	&	Quantitative structure activity relationships	\\
RDF	&	Radial distribution function	\\
RESP	&	Restrained electrostatic potential	\\
RI	&	Resolution of the identity	\\
RMSD	&	Root mean squared deviation	\\
RMSE	&	Root mean squared error	\\
\end{tabularx}
\end{table}
\newpage
\begin{table}[h!]
\def\arraystretch{2.0}
\begin{tabularx}{\textwidth}{YY}
Abbreviation & Meaning \\
\hline
RR	&	Ridgid rotor	\\
SMILES	&	Simplified molecular-input line-entry system	\\
SSD	&	Sum squared distance	\\
STO	&	Slater-type orbtial	\\
TS	&	Transition state	\\
TSA	&	Transition state analogue	\\
VRI	&	Valley-ridge inflection point	\\
WF	&	Wave function	\\
ZPE	&	Zero point energy	\\
\end{tabularx}
\end{table}
\clearpage
\end{document}
