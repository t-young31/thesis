\documentclass[../../main.tex]{subfiles}
\begin{document}
\setcounter{footnote}{0} 


\chapter{autodE}

\section{Overview}


Mechanistic elucidation has played a central role in the development of chemistry, including discovering and optimising chemical reactions and the design of catalysts. For example, new designs are suggested to reduce an energetic barrier to enhance activity, or increase the energy difference between two transition states and, perhaps, enhance selectivity. While there are systems where computation is challenging (e.g. the Morita Baylis–Hillman reaction\cite{Plata2015}), it computational investigations of reaction mechanisms are a useful tool.

Introducing automation may significantly accelerate reaction mechanism elucidation and enable the faster development of novel catalysts. Towards this goal, \emph{autodE} was developed. For a known set of reactants, intermediates and products,  \emph{autodE} targets the construction of reaction (free) energy profiles, enabling the simultaneous exploration of different systems and chemical pathways. The following sections describe the implementation and applications of this tool to a wide range of reaction mechanisms. It also outlines recent developments to accelerate the workflows of computational chemists beyond the examples described in this work. \emph{autodE} will -- and has -- accelerate(d) the workflows of computational chemists. Indeed, other projects have made use of the accelerations possible. Indeed, several collaborative projects used \emph{autodE} to find transition states faster than would otherwise be possible. These include an unusual ring contraction from a 7-membered heterocycle,\cite{Wang2020} diphosphane metathesis via a radical chain process\cite{Branfoot2021} and locating TSs within the final manuscript presented in this thesis.\cite{Young2021GAP}


\clearpage
\end{document}