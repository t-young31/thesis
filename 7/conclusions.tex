\documentclass[../main.tex]{subfiles}
\begin{document}

\setcounter{footnote}{0} 


\chapter{Conclusions}

\section{Modelling Catalytic Processes}

How do we make better catalysts? Certainly computation has a role to play in the discovery and development of existing catalysts; chemical space is simply too large for the experimental alternative. Although the ultimate goal -- from a reaction suggesting a catalyst -- remains elusive, employing calculations to understand and even screen possible catalysts is achievable. This thesis focused on the latter and achieves accelerations that will enable faster predictions.
\\\\
Metallocage modelling provided a fruitful platform to explore and develop techniques to calculate catalytic proficiency within a small subspace relatively quickly. Here, by using potential rather than free energies, a cheap optimisation+single point energy evaluation, and transition state analogues, the catalytic proficiency of [Pd${}_2$L${}_4$]${}^{4+}$ metallocages could be calculated in a few hundred CPUh. To reduce the time-consuming setup time of those calculations \emph{cgbind} was developed to automate the process and lead to an open-source Python module and web-app. Accurate metallocage geometries ($\sim1$ \AA${}$ RMSD to crystal structures) are obtained in a seconds, provided a similar template is available. Rapidly generated features including the electrostatic potential surface and substrate addition provides experimental colleagues to rapidly prototype cages prior to synthesis. Opportunities do however remain; predicting binding affinities of a test set of $\sim 100$ cage-substrate combinations yielded accuracy only just better than a random assignment (60\%). In a similar way to proteins, rigid docking of a substrate is not, generally sufficient to predict binding.\cite{Kitchen2004} In addition to the methodology for predicting turnover, in the [Pd${}_2$L${}_4$]${}^{4+}$ catalysis of Diels-Alder reactions computation also uncovered an un-intuitive flexibility that, at least partially, allows for catalysis. 
\\\\
With a view to automate and accelerate the human-intensive search for conformational global minima and transition states that comprise catalytic cycles \emph{autodE} was developed as an open-source and extensible approach to the problem. From SMILES strings as 1D molecular representations, 3D geometries are built using either a semi-deterministic, fully random build algorithm or an RDKit\cite{Landrum2019} method, aligned for a reaction then a low-energy path traversed from which a TS guess is obtained, optimised and the conformational space searched. Amongst a range of examples, cobalt and rhodium hydroformylation cycles were presented and demonstrate the capability of \emph{autodE} to calculate catalytic cycles. In combination with \emph{molfunc} another open-source tool for molecular functionalisation where 1D/3D molecular fragments are added to a core structure, enabled the optimisation of the catalysts' ligand environment. 
\\\\
Other computational tools were developed for kinetic modelling (\emph{rksim}) which enabled an improved understanding of a [Fe${}_4$L${}_8$]${}^{n+}$ catalysed hydrolysis reaction and (\emph{otherm}) to calculate ideal gas model (and extensions thereof) free energies. The latter was used to calculate more accurate free energies in conjunction with a novel method for calculating solution-phase translational entropy. The method, while simple, is more faithful to the true environment and by relying on a classical treatment of the partition function is efficient to calculate.
\\\\
Towards faster and more accurate computational methods to study realistic (large/solvated) systems a method of training Gaussian Approximation Potentials was developed. The active-learning based approach enables an efficient ML potential to be obtained in a day and is able to accurately propagate dynamics at ms per timestep rather than hours, in some cases. These potentials and more general ones like it will certainly make computational predictions of catalyst behaviour: dynamics, activity, selectivity more efficient and hopefully more accurate.

% MD bad ML better


\section{Future Directions}

The future of computational catalyst design will rely on enhanced experimental collaboration and verification of computational predictions. An increasingly inverse approach to the problem will be required, alongside the development of more, accurate computational methods for energy and force evaluations. Employing a two-pronged strategy to both discover more diverse catalysts and by enumeration of human-guided libraries may be a fruitful approach. Machine learned models trained in a supervised way will continue to dominate the field and drive rapid predictions with perhaps a focus on reinforcement and `self-play' strategies in the future. Increased use of automated methods both computational and synthetic will help to `close-the-loop' and achieve highly effective catalysts for a specific transformation.






\clearpage
\end{document}