\documentclass[../../main.tex]{subfiles}
\begin{document}
\setcounter{footnote}{0} 


\chapter{autodE}

\section{Overview}

The selection of a catalyst for a particular transformation often involves two stages: identification of a initial `hit', then optimisation. Approaches to target both stages computationally are currently being developed, with an end-to-end approach some way off. Mechanistic elucidation is possible using current computational approaches (of course not in all scenarios e.g. ref. \cite{Plata2015}) and can be used to guide either human or computational optimisation by, for example increasing the difference between two TSs enhancing selectivity or simply reducing an energetic barrier to enhance activity.

Introducing automation in computational modelling of mechanisms may significantly accelerate this process, and enable the faster development of novel catalysts. For that reason, \emph{autodE} has been developed and targets the construction of reaction free energy profiles for known connectivity of reactants/intermediates/products. The following sections describe the implementation and some recent developments to the code, which will -- and has -- accelerate(d) the workflows of computational chemists. Indeed, other projects have made use of the accelerations possible (including refs. \cite{Wang2020}, \cite{Branfoot2021}, \cite{Young2021GAP})


\clearpage
\end{document}