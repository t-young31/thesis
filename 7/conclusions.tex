\documentclass[../main.tex]{subfiles}
\begin{document}

\setcounter{footnote}{0} 


\chapter{Conclusions}

\section{Modelling Catalytic Processes}

How do we make better catalysts? Undoubtedly, computation has a central role to play in the discovery and development of existing catalysts; chemical space is simply too large for the experimental alternative. Although the ultimate goal of identifying the optimal catalyst for a given transformation remains elusive, employing calculations has already allowed chemists to understand, and even screen possible catalysts. This thesis focused on the latter and proposed methods that will enable faster predictions.

Modelling metallocage catalysis in Chapter 3 provided the motivation develop techniques to calculate catalytic proficiency within a small subspace relatively quickly. Using potential rather than free energies and transition state analogues, the catalytic proficiency of [Pd${}_2$L${}_4$]${}^{4+}$ metallocages could be calculated in a few hundred CPUh. Computation uncovered an unintuitive flexibility that, at least partially, allows for catalysis.

To reduce the time-consuming setup time of metallocage calculations \emph{cgbind} was developed as an open-source Python module and web-app. Accurate metallocage geometries ($\sim1$ \AA${}$ RMSD to crystal structures) are obtained in seconds for various architectures. Rapidly generated features including the electrostatic potential surface and substrate addition provides experimental colleagues to rapidly prototype cages prior to synthesis. Opportunities for improvement do, however, remain. For example, predicting binary binding affinities of a test set of $\sim 100$ cage-substrate combinations yielded accuracy only just better than a random assignment (60\%). This suggested that, like proteins, rigid docking of a substrate may not be sufficient to predict binding.\cite{Kitchen2004} 

To automate and accelerate the human-intensive search for conformational global minima and transition states that comprise catalytic cycles \emph{autodE} was developed (Chapter 4) as an open-source and extensible approach to the problem. From SMILES strings 3D geometries are built using either a semi-deterministic, fully random build algorithm or an RDKit\cite{Landrum2019} method. A TS guess is obtained by aligning reactant and traversing a low-energy path after which the conformational space of TSs and minima are searched. The general applicability of \emph{autodE} was illustrated in a range of examples, including cobalt and rhodium hydroformylation cycles. The \emph{molfunc} open-source tool was developed alongside for molecular functionalisation, where 1D/3D molecular fragments are added to a core structure, providing an efficient approach to explore and optimise novel catalysts’ scaffolds.  While completely automated these platforms rely on standard methods to calculate free energies and currently do not account for any dynamic effects that may alter the predicted selectivity.

%Other computational tools were developed for kinetic modelling (\emph{rksim}) which enabled an improved understanding of a [Fe${}_4$L${}_8$]${}^{n+}$ catalysed hydrolysis reaction and (\emph{otherm}) to calculate ideal gas model (and extensions thereof) free energies. The latter was used to calculate 

Towards more accurate free energy estimates, a novel method for calculating solution-phase translational entropy was outlined. The simple method is more faithful to the true environment and, by relying on a classical treatment of the partition function, is efficient to calculate. However, the improvement is only apparent for bimolecular reactions and is broadly a systematic shift making `double differences' unaffected. The latter is generally the applicable domain in catalyst optimisation, where only the difference from a reference value is evaluated.


To study the dynamics of realistic (large/solvated) systems efficiently and with a view to obtain accurate free energy differences a method to train Gaussian Approximation Potentials was developed in Chapter 6. The active-learning based approach enables an efficient machine learnt potential to be obtained in a day and is able to accurately propagate dynamics at milliseconds per timestep rather than minutes or hours. These potentials and general ones like it will certainly make computational modelling and catalyst prediction more efficient and more accurate.

% MD bad ML better


\section{Future Directions}

The future of computational catalyst design will rely on enhanced experimental collaboration and verification of computational predictions. An increasingly inverse approach to the problem will be required, alongside the development of more, accurate computational methods for energy and force evaluations. Employing a two-pronged strategy to both discover more diverse catalysts and by enumeration of human-guided libraries may be an effective approach. Machine learned models trained in a supervised way will continue to dominate the field and drive rapid predictions with perhaps a focus on reinforcement and `self-play' strategies in the future. Increased use of automated methods both computational and synthetic will help to `close-the-loop' and achieve highly effective catalysts for a specific transformation.




\clearpage
\end{document}